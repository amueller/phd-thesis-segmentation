\section{Theoretical Analysis of Instance Level Prediction}\seclabel{intro}
%TODO clean up
%Multiple instance learning is a supervised learning scenario first proposed by \citet{dietterich1997solving}.
%In multiple instance learning, examples are multi sets of instances, called bags, and each bag is assigned a class label.
%It is assumed that each instance has a true label, defined by an underlying concept. The label of a bag is positive, if and only if one of
%the instances it contains is labeled positive. The true label of instances is assumed to be unobserved by the learner who only has access to bag labels.

As we mentioned above the original formulation of multiple instance learning, the goal of the learner
it to predict only the label of bags, and only recently was this extended to predicting
the label of individual instances, which is a much harder problem.
This can also be viewed as learning an instance-level classifier using a form of weak supervision.

Learning to predict instance labels has been attempted in some recent work~\citep{liconvex2010,zhang2002dd}, but
 there has been very little theoretical study when and to which extent this is possible.
In this work, we explore which assumptions about the distributions of instances and labels
are necessary to learn instance-level classifiers from bag labels only.
We provide bounds on the instance label prediction error based on the
bag label prediction error. The bounds can be used to judge a learner when no
instance labeling exists for training or testing.

%If one is interested in classifying instances, such a bound is very important.
There are two applications that we target:
%TODO clarify
\begin{description}
\item[Setting 1:] Bag labels for a validation set are known and instance predictions need to be judged.
\item[Setting 2:] Bag labels for a test set can be bounded using standard learning theoretic bounds and instance
predictions need to be judged.
\end{description}
The first case is very common in practical settings, where cross validation is
often used to set parameters of a predictor. This is particularly important in
support vector machines (SVMs) which are a popular learning method for multiple
instance problems~\citep{andrews2003support}. If no instance labels are
present for training, it is only possible to choose parameters to minimize bag
prediction error, which is suboptimal if one is ultimately interested in
minimizing instance label misclassification. The bounds we provide can also be used to
find the amount of positive instances in positive bags, known as the witness
rate, which is an integral part of established multiple instance
algorithms~\citep{zhang2002dd,liconvex2010}. The second case can be used to
convert established bounds on the bag prediction error to instance prediction
error.

%FIXME what do we actually show?
We show that it is possible to bound instance prediction error without
ever observing any instance labels even in fairly unrestricted settings.
To the authors knowledge, this is the first result of this kind.
%
Detailed proofs of all propositions can be found in the appendix.

\subsection{Previous Analysis}
The theoretical analysis of multiple instance learning started with
\citet{auer1997approximating} and later \citet{blum1998note}.  These works
assume that instances are drawn i.i.d.\ from a single distribution over
instances. Blum and Kalai showed that in this setting, the underlying concept
is PAC-learnable in polynomial time, if and only if the multiple instance
learning problem is PAC-learnable in polynomial time with one-sided noise.
\citet{auer1997approximating} showed that if there is an algorithm to PAC-learn
multiple instance learning with axis-parallel rectangles that is polynomial in
size of the bag and dimension of the instance features, then it is possible to
polynomially PAC-learn DNF-formulars, a problem that is solvable only if
$\text{RP}=\text{NP}$.
Recently, \citet{sabato2009homogeneous,DBLP:journals/corr/abs-1107-2021} proved
bounds on the VC dimension and fat shattering dimension of multiple instance
learners by the corresponding dimension for a base learner.

All of the results above only consider predicting bag labels. There has been
very little theoretical work on predicting instance labels. In spite of the
large amount of research on multiple instance learning algorithms (for example
\citet{andrews2003support,gaertner2002multi,zhou2009multi,li2009convex,zhang2002dd,%
mangasarian2008multiple,leistner2010miforests,chen2006miles})
to the authors knowledge, the only quantitative evaluations of predicting
instance labels are by \citet{gehler2007deterministic} and by
\citet{liconvex2010}. While these works show promising results, no analysis of
instance label prediction is given.

\citet{sabato2010reducing} also studied multiple instance learning as a way to
reduce labeling complexity.  In the setting of reducing label complexity, the
learner creates the bags itself. The instances can therefore be assumed to be
globally i.i.d.\ and the distribution of labels inside the bags is known a
priori.
Under these assumptions, the instance label prediction error is bounded using
the bag label error.  This result is used to evaluate the sample complexity of
learning from bags using empirical risk minimization.  In this chapter, we
generalize some results from \citet{sabato2010reducing} to the setting where
the distribution of instance labels is unknown and to cases where instances
inside a bag may be dependent.

\subsection{Preliminaries}
Throughout this chapter, $\mathcal{X}$ denotes the space of all instances. Bags
are multi sets of instances $X=\{x_1,\dotsc, x_r\}$.  As usual in supervised
learning tasks, we assume our training examples ($X_i,Y_i)$ to be i.i.d.\ from
an unknown distribution.  We denote the label of an instance $x_i$ by $y_i$ and
the label of the bag $X_i$ as $Y_i = \max\{y_j | x_j \in  X_i\}$.  We assume
the bag size $r\in \mathbb{N}$ to be constant over all bags in a given problem.
This restriction simplifies the treatment significantly while our results can
in most cases be carried over to settings with varying bag sizes.  We consider
arbitrary classifiers $h_I\colon \mathcal{X} \rightarrow \{0, 1\}$ with
associated bag classifier $h_B(X) = \max \{h(x_1),\dotsc,h(x_r)\}$.  The goal
is to learn a classifier $h$ which minimizes expected instance
misclassification, given training bags and bag labels $X_i$ and $Y_i$.  We do
not consider specific hypothesis classes but instead try to bound the expected
error of $h$ using the expected error of $h_B$.  During all of this chapter, we
assume that the true instance labels are never observed.

\subsection{Necessary Assumptions}\seclabel{assumptions}
If we allow instances inside bags and their labels to have arbitrary
dependencies, we are trying to find $y_1,\dotsc,y_r$ maximizing $p(y_1,\dotsc,
y_r | x_1, \dotsc, x_r)$. This is a structured prediction problem for which
there is no hope solving it, given only bag labels. Therefore, as usual in
multiple instance learning, we assume there is an underlying concept for
instances, that is the label of each instance depends only on this instance.
In other words, we assume $p(y_1,\dotsc, y_r | x_1, \dotsc, x_r)=p(y_1|x_1)
\cdot p(y_r|x_r)$.

If we allow arbitrary dependencies between the instances $x_i$ in a given bag,
there is a simple example \citep{sabato2009homogeneous} that shows that
learning instance classification is not necessarily possible. Let
$\mathcal{X}$ consist of two distinct points $x_1$ and $x_2$ with corresponding
labels $y_1 = 1$ $y_2 = -1$. If the distribution of instances inside bags is
such that each bag either contains two copies of $x_1$ or both of $x_1$ and
$x_2$, then by predicting all bags as positive, one can obtain a perfect bag
classifier. Even though bag classification is perfect, no statement about the
label of $x_2$ can sensibly be made.
Therefore, we need to make additional assumptions about the distribution of
instances inside bags to make predicting instance labels possible.

\subsection{Independent Case}\seclabel{iid}
The simplest case in which learning instance prediction is possible is the case
when instances are globally i.i.d., that is when instances inside a bag are
independent of each other. This is a very strong assumption that might be
violated in practical applications. For example in the above setting of extracting
segments from an image, the segments within a given image are clearly correlated,
and often even overlap.
On the other hand, strong results can be proved using the i.i.d.\ assumption.

\begin{theorem}\label{basicthm}
Let all instances be independent and identically distributed. Let $X$ denote bags, $x$ instances, $y$ true instance labels, and $h_I$ any classifier
on instances. Let $h_B$ the classifier on bag labels that is induced by $h_I$, that is
\[
    h_B(X) = 1 - \prod_{x \in X} (1 - h_I(x)) =
    \begin{cases}
        1 \text{\quad if\quad} h_I(x) = 1\ \forall x \in X\\
        0 \text{\quad else}
    \end{cases}
\]
The we have:
\begin{align*}
    %\eqlabel{iid}
   \instanceerror = \hzero + p(y=0) - 2 \left (\frac{1}{2} ( \hzero^r + p(y=0)^r - \bagerror) \right)^\frac{1}{r}.
\end{align*}
\end{theorem}
This is a generalization of Theorem 1 in \citet{sabato2010reducing} and serves as a basis for all further bounds. 
In particular:

\begin{corollary}\label{perfect}
If all instances are independent and bag classification is perfect, meaning $\bagerror=0$, then instance classification is also perfect ($\instanceerror =0$).
\end{corollary}
This is a very interesting result in that if we do not assume independence, this corollary is wrong,
as shown by the example in \Secref{assumptions}.
%
In practice, $p(y=0)$ is unknown, as we assume a setting where no instance labels are available.
Thus finding a bound independent of $p(y=0)$ is desirable. The following theorem accomplishes this:

\begin{theorem}\label{iidbound}
If $\hzero^r  \geq \bagerror$ and all instances are independent, we have
\begin{align}
    p(h_I(x)\neq y) \leq \hzero - \left ( \hzero^r - \bagerror \right ) ^ \frac{1}{r}
\end{align}
\end{theorem}

This result, which applies to the Setting $2$ from \Secref{intro},
can be used in several ways. First, it can give a general idea about the quality of instance-level
predictions, without assuming any knowledge of the distribution of instance labels. In this inequality,
$\hzero$ can simply be read off on training and/or test data. The other quantity, $\bagerror$ can be
estimated using standard supervised generalization bounds.

If we assume Setting $1$ from \Secref{intro}, where bag labels are known, we can say even more:
\begin{theorem}\label{equality}
Let all instances be independent. Given $p(Y)$, the instance error $\instanceerror$ can be calculated by
\begin{align}
    %\eqlabel{iid}
\instanceerror = \hzero + p(Y=0)^\frac{1}{r} - 2 \left (\frac{1}{2} ( \hzero^r + p(Y=0) - \bagerror) \right)^\frac{1}{r}.
\end{align}
\end{theorem}
If bag labels are given, $p(Y=0)$ and $ \bagerror$ can simply be estimated from validation data.
This means, in the i.i.d.\ case, instance-level error can simply be calculated
from bag labels and classifier statistics.

\subsection{Restricted Dependent Case}
As remarked in \Secref{assumptions}, it is not possible to bound the instance classification error
when allowing arbitrary dependencies between instances. On the other hand, the i.i.d.\ assumption of \Secref{iid}
might be too strong in a practical setting.
Therefore, we propose an assumption that is both very general and allows learning of instance classification:
We assume that each bag has a ``hidden cause'' $Z$, conditioned on which all instances are independent.
More formally, we assume
\begin{align}
    p(x_1,\dotsc,x_r)=\int_Z p(x_1)\cdot \dotsc \cdot p(x_r) \, dZ.
\end{align}
The intuition behind this assumption is as follows.
For the above example of object class segmentation, one might imagine the class of the present object,
or even the type of the scene to be this hidden cause. Knowning which class might be present
in an image disentangles the dependency structure of the segments.

Another application with an natural interpretation of the above assumption is
scene classification~\citep{zhou2007multi,zha2008joint,zhou2009multi}, a very
popular application of multiple-instance learning.
In this case, each image is represented as a collection of interest points or
segments.  For the sake of the argument let us assume that interest points or
segments correspond to objects.  Again, objects in a scene are not independent.
For example, if we know an image contains a chair and a TV set, it is very
unlikely to contain a bus.  On the other hand, if we knew something about the
scene, such as ``this is a living room scene'', then the appearance of certain
objects does not tell us much about other objects.

In the classical case of drug activity
prediction~\citep{dietterich1997solving}, a bag corresponds to a molecule and
each instance corresponds to a different geometric configurations. These are
certainly not independent.  If one knew the ``true'' geometric characteristics
of the molecule, each configuration could be derived from that.

The idea of introducing a hidden cause is similar to the idea of viewing a bag
as a manifold and instances as points sampled on this manifold, as in
\citet{ICML2011Babenko_74}.

Conditioned on the hidden cause $Z$ for a given bag, the proof of Theorem~\ref{basicthm} still holds, yielding
\begin{align}\eqlabel{conditioned}
p(h_I(x) \neq y | Z) =&p(h_I(x)=0|Z) + p(y=0|Z)\\
&\textstyle- 2  [\frac{1}{2} ( p(h_I(x)=0|Z)^r + p(y=0|Z)^r
 - p(h_B(X) \neq Y|Z)) ]^\frac{1}{r}
\end{align}
Similarly, the idea of Corollary~\ref{perfect} yields:

\begin{corollary}\label{perfect}
If the bag classification is perfect, meaning $\bagerror=0$, then instance classification is also perfect.
\end{corollary}

While this is a very simple result, is has strong implications. In particular, albeit we generalized our
assumptions significantly, learning instance label prediction is still possible.
As we assume no knowledge of the distribution of the hidden bag variable $Z$ or even about its domain, it is significantly harder
to bound the instance error than in the i.i.d.\ case.
Using Theorem~\ref{iidbound} and taking expected values with respect to $Z$, we obtain
\begin{equation}
    p(h_I(x)\neq y)
    \leq \hzero - \mathbb{E}_Z \left [\left ( p(h_I(x)=0 | Z)^r - p(h_B(X)\neq Y|Z) \right ) ^ \frac{1}{r} \right ].
\end{equation}
Unfortunately, simplifying this expression further requires more assumptions
about the nature of $Z$ which we will not investigate further here.


\begin{figure}[tbp]
	\begin{center}
		\includegraphics[width=.40 \linewidth]{images/gehler_decision_boundary.png}\hspace{35px}
		\includegraphics[width=.40 \linewidth]{images/gehler_instances.png}
	\end{center}
	\caption{Visualization of the dataset from \citet{gehler2007deterministic}. Left: Regions in $\mathbb{R}^2$
    corresponding to negative and positive instances. Right: Training points for 40\% positive rate. Red denotes negative
    bags, blue denotes positive bags. Crosses denote positive instances, circles negative instances.}
	\figlabel{gehler}

\end{figure}
To get a better handle on the behavior of instance errors, it may also be useful to lower-bound $\instanceerror$.
This is possible using Theorem~\ref{basicthm}:

\begin{theorem}\label{lowerbound}
If the instance labels are independent inside a bag given a hidden cause, the instance
error $\instanceerror$ has the following lower bound:
\begin{equation}
\instanceerror \geq \hzero + p(y=0)
- 2 \left (\frac{1}{2} (p(h_B(X)=0) + p(Y=0) - \bagerror) \right)^\frac{1}{r}.
\end{equation}
\end{theorem}


\section{Experiments}
We evaluate the tightness of our bounds and their utility for finding parameters on the sythetic dataset of \citet{gehler2007deterministic}.
This dataset allows us to vary $p(y=0)$, which is half of the positive rate in positive bags.
A visualization of the dataset, which we refer to as Gehler-org is shown in \Figref{gehler}. In Gehler-org, there are as many positive
as negative bags, 30 of which are commonly used for training and 100 of which are used for testing.
We use 1000 bags for testing, resulting in more stable estimates of errors and bounds.
When generating the dataset, one fixes a rate of positive instances, also known as witness rate, which determines the ratio
of positive instances in a positive bag.

To evaluate our bounds in the i.i.d.\ setting, we have to modify the dataset slightly. We fix bag size at ten instances and generate instances
with a fraction of positive instances of half the witness rate. Then we randomly assign these instances to bags.
Using this procedure the distribution over instances stays the same as in Gehler-org while instances inside bags become independent.
We refer to this dataset as Gehler-iid.
As a multiple instance learning algorithm, we choose SVM-SVR from \citet{liconvex2010}, as a state-of-the-art method that
provides instance predictions. In SVM-SVR, support vector regression is learned to estimate likelihood ratios of label instances.

\begin{figure}[tbp]
	\begin{center}
		\includegraphics[height=63mm]{images/iid_01.png}
		\includegraphics[height=63mm]{images/iid_02.png}
		\includegraphics[height=63mm]{images/iid_03.png}
	\end{center}
	\caption{Error rates and bounds for dataset Gehler-iid. Top, center and bottom correspond to positive rates of 0.1, 0.2 and 0.3 respectively.
    We only look at low positive rates as in the i.i.d.\ case, high positive rates result in all bags being positive.
    See the text for details.}
	\figlabel{gehler_iid}
\end{figure}

\Figref{gehler_iid} compares the true instance error $\instanceerror$, the bag error $\bagerror$, the estimate of $\instanceerror$ using Theorem~\ref{equality}
and the bound of Theorem~\ref{iidbound} for datasets with different values of the positive instance rate.
The $x$-axis shows different thresholds on estimated likelihood ratios, resulting in varying $\hzero$.
It is apparent that the estimate of Theorem~\ref{equality} reflects the true error closely, with only little noise from the restricted
sample size. Note that the upper bound of Theorem~\ref{iidbound} is not defined everywhere. While it is quite tight when
$h_I^r \gg \bagerror$, it is not when $h_I^r \approx \bagerror$.


We also performed experiments using a dependent version of Gehler-org
with probabilistic constraints. Ten examples are drawn per bag and each example is positive with probability
of the fixed positive rate. We denote this dataset by Gehler-soft. In the original dataset, the number of positives and negatives in a positive bag
was deterministic, making the instances strongly dependent.

The new dataset Gehler-soft violates the i.i.d.\ assumption but satisfies the
restricted dependent assumption. \Figref{gehler_soft}
shows the values the lower bound from Theorem~\ref{lowerbound} compared to the
true instance error $\instanceerror$ for different settings of the positive instance rate.
Inspecting the lower bound, which uses the same computation as the estimate in
the i.i.d.\ case, we see that due to the limited dataset size, the estimated
lower bound actually lies above the actual error in some cases. On the other
hand, for low positive rates, the lower bound follows the true error quite
closely.

\begin{figure}[tbp]
	\begin{center}
		\includegraphics[height=63mm]{images/non_iid_02.png}
		\includegraphics[height=63mm]{images/non_iid_05.png}
		\includegraphics[height=63mm]{images/non_iid_09.png}
	\end{center}
	\caption{Error rates and bounds for dataset Gehler-soft. Top, center and bottom correspond to positive rates of 0.2, 0.5 and 0.9 respectively.
    See the text for details.}
	\figlabel{gehler_soft}
\end{figure}

\section{Summary}
We proposed an algorithm for object-class segmentation using only weak supervision based on
multiple-instance learning. In our approach each image is represented as a bag of object-like
proposal segments.

We described a way to extent bag level predictions made by the multi-instance
kernel method to instance level while remaining competitive with the state-of-the-art
in bag label prediction.

We evaluated the proposed object-class segmentation method on the challenging
Graz02 dataset. While not reaching the performance of methods using strong supervision,
our result can work as a baseline for further research into weakly supervised object class
segmentation.

%In future work, we plan to scale our approach to much larger image datasets. As much
%more images with weak annotations are available than with pixel level segmentation,
%we hope that we can improve upon the state-of-the-art in object-class segmentation
%by making use of larger bodies of training images.
On the theoretical side, we provided several bounds on instance prediction
error in a multiple instance learning setting.  For the case that instances
inside a bag are independent, we proved a strong upper bound on the error.  We
showed that when bag labels are present, it is even possible to directly
estimate the true instance error.

We introduced a restricted dependent setting for multiple instance learning, in
which instances inside a bag are only independent given an unobserved variable.
We showed that without any further assumptions on the distribution of this
variable, it is possible to learn instance label prediction from bag labels in
this setting. We also provided a lower bound on the instance error,
given known bag labels or an estimage of the bag error.
Empirical evaluation on a synthetic dataset with known dependency structure and
known instance labels showed that the bound are non-trivial and yield insight
into the behavior of instance label errors.
We hope that this analysis establishes a basis for further research on instance
label prediction in multiple instance learning.
