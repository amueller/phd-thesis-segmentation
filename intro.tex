\chapter{Introduction}
This thesis investigates the use of structured learning algorithms to the task of semantic image
segmentation. Both topics have recieve much attention in the computer vision
and machine learning communities lately. Unfortunately, learning structured
prediction in computer vision applications is still little understood.
We hoe this thesis will further the understanding of structured learning methods,
give practical guidance and provide implementations to further the use of structured
prediction methods in computer vision.

The task of semantic segmentation is defined as predicting a semantic class label
for each pixel in a natural image. Predicting semantic labels on such a fine scale
provides ??? an important step towards general scene understanding.
%
Applications of scene understanding include automatic image interpretation for retrieval,
 autonomous driving and mobile robotics.
%
related work?? here again?
Current systems for semantic segmentation emply a wealth of different paradigms, most notably:
The ranking of figure ground hypotheses
conditional random field / superpixel? kraehenbuehl?
detector based
sparse coding

We will focus on the use of conditional random fields, which have shown very promising.
Using the paradigm of structural support vector machines, it is possible to learn conditional
random fields to directly minimize a loss of interest. In particular, conditional random fields allow
to combine different cues, possibly produced by other approaches, in a principled manner.
%
We use four different datasets to evaluate our approach: Graz02, Pascal VOC 2010, MSRC-21
and NYU V2.
%
Why semantic segmentation? Difference object / stuff centric? -> difference to Alex
Structured approaches
%

\section{List of Contributions}
Contributions:
- Introduces general clustering algorithm that improves pipeline in serveral places.
- Shows that weakly supervised object class segmentation (not semantic segmentation) is possible.
- Provides general efficient framework for structured prediction.
- Analyze max-margin learning algorithms in different regimes.
- shows exact learning is possible in loopy graphs and improves results (sometimes?).
- shows importance of edge features.

\section{Thesis Outline}
We start by introducing a novel clustering algorithm in Chapter~\ref{ch:itm}, which is used as a building
block in several places for our semantic segmentation algorithms.
We will then introducing a mostly unstructured approach to semi-supervised learning for semantic segmentation in Chaper~\ref{ch:semi_supervised},
investigating the problem of annotating training data for semantic segmentation.
The central topic of this thesis is introduced in Chapter~\ref{ch:structured_pystruct}, which also introduces our software library
for implementing structured learning and prediction algorithms.
Chapter~\ref{ch:comparision} gives a quantitative comparison of the most widely used structured prediction algorithms, focusing on
which algorithm is best to use in a given situation. In particular, we investigate learning behavior for semantic segmentation on several datasets.

The problem of learning with approximate inference is investigated in Chapter~\ref{ch:exact_learning}.
We develop a strategy for efficient caching and a combination of inference algorithms that allows us to
learn SSVMs for semantic image segmentation exactly, even though the involved factor graphs contain many loops.

Finally, Chapter~\ref{ch:nyu} applies the methods developed in Chapters~\ref{ch:structurd_pystruct} and \ref{ch:exact_learning}
to the problem of semantic annotation with structure classes in RGB-D data. We demonstrate that we are able to learn
meaningful spacial relations, and improve upon the state of the art in the NYU V2 datasets that we consider.

\section{Publications}
The main material of this thesis has either been published in conference proceedings or has been
submitted to conferences or journals. We now list the relevant publications.
\begin{description}
    \item[Chapter 2] \emph{Information Theoretic Clustering using Minimum Spanning Trees} Andreas C. M\"uller, Sebastian Nowozin and Christoph H. Lampert. Published in the proceedings of the German Conference on Pattern Recognition.
    \item[Chapter 3] \emph{Multi-Instance Methods for Partially Supervised Image Segmentation} Andreas C. M\"uller and Sven Behnke. Published in the proceedings of the IARP Worshop on Partially Supervised Learning.
    \item[Chapter 4] \emph{PyStruct - Structured Prediction in Python} Andreas C. M\"uller and Sven Behnke. Submitted to the Journal of Machine Learning Research, Open Source Software track.
    \item[Chapter 5] \emph{Learning a Loopy Model for Semantic Segmentation Exactly}. Andreas C. M\"uller and Sven Behnke. Submitted to the International Conference on Computer Vision Theory and Applications.
    \item[Chapter 6] \emph{Learning Depth-Sensitive Conditional Random Fields for Semantic Segmentation}. Andreas C. M\"uller and Sven Behnke. Submitted to the International Conference on Robotics and Automation.
\end{description}

