\chapter{Introduction}
The task of semantic segmentation is defined as predicting a semantic class label
for each pixel in a natural image. Predicting semantic labels on such a fine scale
constitutes an important step towards general scene understanding.
%
Applications of scene understanding include automatic image interpretation for retrieval,
autonomous driving and mobile robotics.
%
We will in the following distinguish between the task of semantic segmentation,
which is usually distinguishing unstructured ``stuff'' classes such as road and
grass, and object class segmentation, which denotes the segmentation of very
structured classes, such as cars, planes and people.
We consider four different datasets in this thesis: the object class segmentation
datasets Graz02 and Pascal VOC 2010, and the semantic segmentation datasets MSRC-21,
and NYU V2.
%
Both tasks have the same ultimate goal of parsing, and therefore understanding, images
in terms of semantic classes, but usually employ  different mechanisms to represent the input.
%
%The main part of this thesis investigates the tasks of semantic segmentation and object class segmentation in natural images.
The main part of this thesis investigates the use of structured learning
algorithms to the task of semantic image segmentation. Both topics have receive
much attention in the computer vision and machine learning communities lately.
Unfortunately, learning structured prediction in computer vision applications
is still little understood.  We hoe this thesis will further the understanding
of structured learning methods, give practical guidance and provide
implementations to further the use of structured prediction methods in computer
vision.

%related work?? here again?
%Current systems for semantic segmentation emply a wealth of different paradigms, most notably:
%The ranking of figure ground hypotheses
%conditional random field / superpixel? kraehenbuehl?
%detector based
%sparse coding

We will focus on the use of conditional random fields, which have shown very
promising.  Using the paradigm of structural support vector machines, it is
possible to learn conditional random fields to directly minimize a loss of
interest. In particular, conditional random fields allow to combine different
cues, possibly produced by other approaches, in a principled manner.
%
Why semantic segmentation? Difference object / stuff centric? -> difference to Alex
Structured approaches
%

\section{List of Contributions}
\begin{itemize}
\item Introduction of general clustering algorithm that improves upon widely used
    approaches from the literature.
\item Demonstration of an algorithm for  weakly supervised object class segmentation.
\item Providing a general efficient implementation for structured prediction.
\item Analysis of max-margin learning algorithms with exact or approximate inference in different applications.
\item Demonstrating that exact learning for semantic segmentation and object class segmentation is possible in loopy graphs.
%\item Shows importance of edge features.
\end{itemize}

\section{Thesis Outline}
We start by introducing a novel clustering algorithm in Chapter~\ref{ch:itm},
which we plan to use as an additional building block for semantic segmentation
algorithms in future work. Clustering is important both for the initial
segmentation into superpixels, and in the creation of visual words.

We will then introducing a mostly unstructured
approach to semi-supervised learning for object class segmentation in
Chaper~\ref{ch:semi_supervised}, investigating the problem of annotating
training data for semantic segmentation. The central topic of this thesis is
introduced in Chapter~\ref{ch:structured_pystruct}, which also introduces our
software library for implementing structured learning and prediction
algorithms.
Chapter~\ref{ch:comparision} gives a quantitative comparison of the most widely
used structured prediction algorithms, focusing on which algorithm is best to
use in a given situation. In particular, we investigate learning behavior for
semantic segmentation on several datasets.

The problem of learning with approximate inference is investigated in
Chapter~\ref{ch:exact_learning}.  We develop a strategy for efficient caching
and a combination of inference algorithms that allows us to learn SSVMs for
semantic image segmentation exactly, even though the involved factor graphs
contain many loops. We demonstrate our algorithm on the Pascal VOC 2010 and MSRC-21 datasets,
and provide state-of-the-art results on the MSRC-21 dataset.

Finally, Chapter~\ref{ch:nyu} applies the methods developed in
Chapter~\ref{ch:structurd_pystruct} and Chapter~\ref{ch:exact_learning} to the
problem of semantic annotation with structure classes in RGB-D data. We
demonstrate that we are able to learn meaningful spacial relations, and improve
upon the state-of-the-art in the NYU V2 datasets that we consider.

\section{Publications}
The main material of this thesis has either been published in conference
proceedings or has been submitted to conferences or journals. We now list the
relevant publications.
\begin{description}
    \item[Chapter 2] \emph{Information Theoretic Clustering using Minimum Spanning Trees} Andreas C. M\"uller, Sebastian Nowozin and Christoph H. Lampert. Published in the proceedings of the German Conference on Pattern Recognition.
    \item[Chapter 3] \emph{Multi-Instance Methods for Partially Supervised Image Segmentation} Andreas C. M\"uller and Sven Behnke. Published in the proceedings of the IARP Worshop on Partially Supervised Learning.
    \item[Chapter 4] \emph{PyStruct - Structured Prediction in Python} Andreas C. M\"uller and Sven Behnke. Submitted to the Journal of Machine Learning Research, Open Source Software track.
    \item[Chapter 5] \emph{Learning a Loopy Model for Semantic Segmentation Exactly}. Andreas C. M\"uller and Sven Behnke. Submitted to the International Conference on Computer Vision Theory and Applications.
    \item[Chapter 6] \emph{Learning Depth-Sensitive Conditional Random Fields for Semantic Segmentation}. Andreas C. M\"uller and Sven Behnke. Submitted to the International Conference on Robotics and Automation.
\end{description}
