\chapter{Datasets and Evaluation Methodology}
This work uses several public datasets for semantic segmentation
and object class segmentation. This chapter briefly introduces each of the datasets,
and discusses its characteristic. The datasets we use are among the most well-known
and most-used datasets in the current semantic segmentation and object class segmentation
literature.

\section{Graz}
% really?

\section{Microsoft Research Cambridge Segmentation}
% intro + description
The Microsoft Research Cambridge dataset V2 (MSRC-23) was introduced in \citet{shotton2006textonboost}, and
has been widely used since. It is a dataset for semantic segmentation, and contains annotations for
several object categories: car, body, face, dog, boat, plane, bike, cow, cat, sheep,
building, bird, book, chair, horse, and sign. It also contains several ``stuff''
categories: sky, grass, tree, water, road, flower, and mountain.
This division is somewhat arbitrary, as some of the object classes also have a ``stuff'' character:
buildings are often not isolated, but rather whole street-scenes, and books are mostly whole
shelfs.
%
As the dataset contains only very few instances of mountain and horse, we do
not consider these classes, as is common in the literature. The resulting
dataset is known as MSRC-21.
The dataset contains %FIXME
images in total, usually split into a training set of size , a validation set of size ,
and a testing set of size .

% characteristics
The size of the dataset is quite small considering the variety of categories. In particular the
bird class consists mostly of ducks, swans and geese, but also contains two instances of peacocks
and one image showing tits. Few samples combined with large intra-class variety introduces considerable
variance in the performance. Using a fixed training and test set makes comparison with the literature possible,
but does not solve this problem.
%
On the other hand, many of the images obey very similar scene layouts, such as centered cars from the side,
centered planes on the run way, and centered faces. We will keep with the literature in not exploiting
this property of the dataset, as it can be considered an unwanted artifact of the dataset collection.
One striking characteristic of the MSRC dataset is that 

Most of the classes in the MSRC-21 dataset can be well distinguished using local texture.
This is particularly true for the stuff classes, but also for the animal classes, buildings, and signs.
Consequently, many approaches use low-level approaches such as texton-boost or random forests as base features,
as these are particularly well-suited for the dataset.


\section{Pascal VOC}

\section{New York University RGB-D Segementation 

\section{Evaluation}
